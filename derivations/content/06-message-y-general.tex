\section{Message $\nu(y)$}

The message $\nu(y)$ can be calculated as
\begin{equation}
    \ln \nu(y) = \mathrm{E}_{q(\bm{x})}\left[\ln p(y\mid x_1, \ldots, x_K)\right]
\end{equation}
\begin{equation}
    \begin{split}
        \ln \nu(y) 
        &= \mathrm{E}_{q(\bm{x})}\left[\ln p(y\mid x_1, \ldots, x_K)\right] + \textit{const} \\
        &\approx \mathrm{E}_{q(\bm{x})}\left[-\frac{1}{\sum_{k=1}^K \exp(m_{x_k})}|y|^2 \right. \\
        &\qquad \left.+(\exp.(\bm{m}_x) \circ \sigma(\bm{m}_x) \circ \sigma(\bm{m}_x))^\top (\bm{x}-\bm{m}_x)|y|^2\right] + \textit{const} \\
        &= -\frac{1}{\sum_{k=1}^K \exp(m_{x_k})}|y|^2  +\textit{const}
    \end{split}
\end{equation}
From this description the variational message can be identified as   
\begin{equation}
    \boxed{
        \nu(y) \propto \mathcal{N}_\mathcal{C}\left(y \ \bigg\vert \ 0,\ \sum_{k=1}^K \exp \left( m_{x_k}\right) ,\ 0 \right)
    }
\end{equation}